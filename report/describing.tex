\subsection{Connected Data Across Views \href{http://randy.cs.columbia.edu/lineage/pgbench-connect/pgbench.html}{(\underline{link})}}\label{connect}
\begin{figure}[H]
	\includegraphics[width=\columnwidth]{figures/connect}
	\caption{A screenshot of the visualizations described in \autoref{connect}
	}
	\label{fig_connect}
\end{figure}
\subsubsection{Vega}
\subsubsection*{Scatter Plot ($V_1$, left view in Figure~\ref{fig_connect})}
\lstinputlisting[language=json, tabsize=2, emptylines=0]{../pgbench-connect/vega/scatterplot.json}
\subsubsection*{Bar Chart ($V_2$, right view in Figure~\ref{fig_connect})}
\lstinputlisting[language=json, tabsize=2, emptylines=0]{../pgbench-connect/vega/barchart.json}
\subsubsection{Succinct Event-driven Description (SED) \#1 (whiteboarded on 11/12)}
Let $V_1$ be the scatter plot.
Let $V_2$ be the bar chart.
\begin{align*}
	V_1&= \mathlarger{\pi}_{circle}(\bf{accounts})\\
	S_1&= \mathlarger{\pi}_{circle}(\mathlarger{\sigma}_{P_1}(\bf{accounts}))\\
	V_2&= \mathlarger{\pi}_{rect}(\mathlarger{\gamma}_{ccolor, \text{SUM}(abalance)}(\bf{accounts}\bowtie{}\bf{tellers}))\\
	S_2&= \mathlarger{\pi}_{rect}(\mathlarger{\gamma}_{ccolor, \text{SUM}(abalance)}(\mathlarger{\sigma}_{P_2}(\bf{accounts}\bowtie{}\bf{tellers})))\\
	init()&:\\
	reset()&:\\
	&: P_1 = P_2 = *\\
	&: S_1.color = S_2.color = \texttt{steelblue}\\
	click(V_2)&:\\
	&: P_2 = ``\text{WHERE}~ccolor=click.ccolor"\\
	&: P_1 = ``\text{WHERE}~aid~\text{IN~lineage}(S_2, \bf{accounts})"\\
	&: S_1.color = S_2.color = \texttt{red}\\
\end{align*}
\subsubsection{SED \#2}
The same interaction can be described using a single predicate for both views.
\begin{align*}
	V_1&= \mathlarger{\pi}_{circle}(\bf{accounts})\\
	S_1&= \mathlarger{\pi}_{circle}(\mathlarger{\sigma}_{P_2}(\bf{accounts}\bowtie{}\bf{tellers})))\\
	V_2&= \mathlarger{\pi}_{rect}(\mathlarger{\gamma}_{ccolor, \text{SUM}(abalance)}(\bf{accounts}\bowtie{}\bf{tellers}))\\
	S_2&= \mathlarger{\pi}_{rect}(\mathlarger{\gamma}_{ccolor, \text{SUM}(abalance)}(\mathlarger{\sigma}_{P_2}(\bf{accounts}\bowtie{}\bf{tellers})))\\
	init()&:\\
	reset()&:\\
	&: P_2 = *\\
	&: S_1.color = S_2.color = \texttt{steelblue}\\
	click(V_2)&:\\
	&: P_2 = ``\text{WHERE}~ccolor=click.ccolor"\\
	&: S_1.color = S_2.color = \texttt{red}\\
\end{align*}
\subsubsection{SED \#3 (expanded to include clicks on $V_1$)}\label{connect_3}
\begin{align*}
	V_1&= \mathlarger{\pi}_{circle}(\bf{accounts})\\
	S_1&= \mathlarger{\pi}_{circle}(\mathlarger{\sigma}_{P_1}(\bf{accounts}))\\
	V_2&= \mathlarger{\pi}_{rect}(\mathlarger{\gamma}_{ccolor, \text{SUM}(abalance)}(\bf{accounts}\bowtie{}\bf{tellers}))\\
	S_2&= \mathlarger{\pi}_{rect}(\mathlarger{\gamma}_{ccolor, \text{SUM}(abalance)}(\\&\mathlarger{\sigma}_{P_2}(\bf{accounts}\bowtie{}\bf{tellers})))\\
	init()&:\\
	reset()&:\\
	&: P_1 = P_2 = *\\
	&: S_1.color = S_2.color = \texttt{steelblue}\\
	select(V_1)&:\\
	&: reset()\\
	&: P_1 = ``\text{WHERE}~aid~\text{IN}~\mathlarger{\pi}_{\bf{accounts}}(select)"\\
	&: P_2 = ``\text{WHERE}~ccolor\text{IN~SELECT~DISTINCT}~ccolor~\\&\text{FROM}~\mathlarger{\pi}_{\bf{accounts}}(S_1)\bowtie{}\bf{tellers}"\\
	&: S_1.color = S_2.color = \texttt{red}\\
	click(V_2)&:\\
	&: reset()\\
	&: P_2 = ``\text{WHERE}~ccolor=click.ccolor"\\
	&: P_1 = ``\text{WHERE}~aid~\text{IN~lineage}(S_2, \bf{accounts})"\\
	&: S_1.color = S_2.color = \texttt{red}\\
\end{align*}
\subsection{Aggregation Filtering \href{http://randy.cs.columbia.edu/lineage/pgbench-filter/pgbench.html}{(\underline{link})}}\label{filter}
\begin{figure}[H]
	\includegraphics[width=\columnwidth]{figures/filter}
	\caption{A screenshot of the visualizations described in \autoref{filter}
	}
	\label{fig_filter}
\end{figure}
\subsubsection{Vega}
\subsubsection*{Satisfaction Bar Chart ($V_3$, left view in Figure~\ref{fig_filter})}
\lstinputlisting[language=json, tabsize=2, emptylines=0]{../pgbench-filter/vega/satisfactionbars.json}
\subsubsection*{Balance Bar Chart ($V_4$, right view in Figure~\ref{fig_filter})}
\lstinputlisting[language=json, tabsize=2, emptylines=0]{../pgbench-filter/vega/balancebars.json}
\subsubsection{SED}
Let $V_3$ be the satisfaction bar chart.
Let $V_4$ be the balance bar chart.
\begin{align*}
	V_3&= \mathlarger{\pi}_{rect}(\mathlarger{\gamma}_{bid, \text{AVG}(satisfaction)}(\bf{accounts}\bowtie{}\bf{tellers}))\\
	S_3&= \mathlarger{\pi}_{rect}(\mathlarger{\gamma}_{bid, \text{AVG}(satisfaction)}(\mathlarger{\sigma}_{P_3}(\bf{accounts}\bowtie{}\bf{tellers})))\\
	V_4&= \mathlarger{\pi}_{rect}(\mathlarger{\gamma}_{ccolor, \text{SUM}(abalance)}(\bf{accounts}\bowtie{}\bf{tellers}))\\
	S_4&= \mathlarger{\pi}_{rect}(\mathlarger{\gamma}_{ccolor, \text{SUM}(abalance)}(\mathlarger{\sigma}_{P_4}(\bf{accounts}\bowtie{}\bf{tellers})))\\
	init()&:\\
	reset()&:\\
	&: P_3 = P_4 = *\\
	&: S_3.color = S_4.color = \texttt{steelblue}\\
	click(V_4)&:\\
	&: P_3 = P_4 = ``\text{WHERE}~ccolor=click.ccolor"\\
	&: S_3.color = S_4.color = \texttt{red}\\
\end{align*}
\subsection{Filter \& Elaborate}
Consider an amendment to the visualization in \autoref{connect_3} wherein the bar chart $V_2$ is replaced by a map showing the branch locations $V_2'$.
On mousing over any of the branch locations, all colors worn by any of the tellers at that branch are shown in a flyout.
When some accounts are selected from the scatterplot, only the tellers who manage those accounts are considered.
\subsubsection{SED}
Let $V_1$ be the scatter plot.
Let $V_2'$ be the map.
\begin{align*}
	V_1&= \mathlarger{\pi}_{circle}(\bf{accounts})\\
	S_1&= \mathlarger{\pi}_{circle}(\mathlarger{\sigma}_{P_1}(\bf{accounts}))\\
	V_2'&= \mathlarger{\pi}_{star}(\mathlarger{\gamma}_{ccolor, \text{SUM}(abalance)}(\bf{accounts}\bowtie{}\bf{tellers}))\\
	S_2'&= \mathlarger{\pi}_{ccolor}(\mathlarger{\sigma}_{P_2'}(\bf{accounts}\bowtie{}\bf{tellers}\bowtie{}\bf{branches}))\\
	init()&:\\
	reset()&:\\
	&: P_1 = *\\
	&: S_1.color = \texttt{steelblue}\\
	select(V_1)&:\\
	&: reset()\\
	&: P_1 = ``\text{WHERE}~aid~\text{IN}~\mathlarger{\pi}_{\bf{accounts}}(select)"\\
	&: S_1.color = \texttt{red}\\
	mouseover(V_2)&:\\
	&: reset()\\
	&: P_2' = ``\text{WHERE}~aid~\text{IN}~\mathlarger{\pi}_{\bf{accounts}}(S_1)~\text{AND}\\&bid=\mathlarger{\pi}_{\bf{branches.bid}}(mouseover.star)"\\
	&: pointer.flyout = \mathlarger{\pi}_{textbox}(S_2')\\
\end{align*}
\subsection{Connect (multiple tellers per account)}
Consider an amendment to the visualization in \autoref{connect_3} wherein each account is managed by two tellers.
In order to associate account records with teller records, a join table named \textbf{accounts\_join\_tellers} is used.
\subsubsection{SED}
\begin{align*}
	V_1&= \mathlarger{\pi}_{circle}(\bf{accounts})\\
	S_1&= \mathlarger{\pi}_{circle}(\mathlarger{\sigma}_{P_1}(\bf{accounts}))\\
	V_2&= \mathlarger{\pi}_{rect}(\mathlarger{\gamma}_{ccolor, \text{SUM}(abalance)}(\\&\bf{accounts}\bowtie{}\bf{accounts\_join\_tellers}\bowtie{}\bf{tellers}))\\
	S_2&= \mathlarger{\pi}_{rect}(\mathlarger{\gamma}_{ccolor, \text{SUM}(abalance)}(\\&\mathlarger{\sigma}_{P_2}(\bf{accounts}\bowtie{}\bf{accounts\_join\_tellers}\bowtie{}\bf{tellers})))\\
	init()&:\\
	reset()&:\\
	&: P_1 = P_2 = *\\
	&: S_1.color = S_2.color = \texttt{steelblue}\\
	select(V_1)&:\\
	&: reset()\\
	&: P_1 = ``\text{WHERE}~aid~\text{IN}~\mathlarger{\pi}_{\bf{accounts}}(select)"\\
	&: P_2 = ``\text{WHERE}~ccolor\text{IN~SELECT~DISTINCT}~ccolor~\\&\text{FROM}~\mathlarger{\pi}_{\bf{accounts}}(S_1)\bowtie{}\bf{accounts\_join\_tellers}\bowtie{}\bf{tellers}"\\
	&: S_1.color = S_2.color = \texttt{red}\\
	click(V_2)&:\\
	&: reset()\\
	&: P_2 = ``\text{WHERE}~ccolor=click.ccolor"\\
	&: P_1 = ``\text{WHERE}~aid~\text{IN~lineage}(S_2, \bf{accounts})"\\
	&: S_1.color = S_2.color = \texttt{red}\\
\end{align*}
\subsubsection{Analysis}
This interaction can be indexed with a data cube just as in \autoref{connect_3}.
\subsection{Vega vs. Alternatives}
\underline{Similarities}:
\begin{enumerate}
	\item Both descriptions are partially declarative and partially stateful.
	\item Both descriptions are event-driven, with state changes propagating up from interaction events to modify intermediate state, and eventually the rendered views (if necessary).
	\item The tables involved in rendering each view are immediately apparent from the description of the view/marks.
\end{enumerate}
\underline{Differences}:
\begin{enumerate}
	\item The alternate description is far more succinct than Vega.
	\item The alternate description leaves out many details.
		For example:
		\begin{itemize}
			\item scaling and position details contained in the projections $\mathlarger{\pi}_{circle}$ and $\mathlarger{\pi}_{rect}$;
			\item descriptive names for scales, predicates, and marks; and
			\item scaling, position, and signal details corresponding to interaction events (this makes it difficult to specify predicates which depend on those events).
		\end{itemize}
\end{enumerate}
\subsection{Possible Indicators That Lineage Can Be Used for Optimization}
\begin{enumerate}
	\item The same tables are seen in the definitions for two different selection symbols.
	\item The predicate used to define one selection symbol contains another selection symbol.
	\item Two selection symbols use the same predicate.
\end{enumerate}
\subsection{Possible Indicators That Data Cubes Can Be Used for Optimization}
\begin{enumerate}
	\item The same exact mappings of the same tables are seen in the definitions for two different selection symbols.
\end{enumerate}