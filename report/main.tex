\documentclass[fleqn]{sigplanconf}
\usepackage{amsmath,listings,xcolor,hyperref,graphicx,float,relsize}

\renewcommand{\sectionautorefname}{\S}
\renewcommand{\subsectionautorefname}{\S}
\renewcommand{\subsubsectionautorefname}{\S}

% from http://tex.stackexchange.com/a/83100
\colorlet{punct}{red!60!black}
\definecolor{background}{HTML}{EEEEEE}
\definecolor{delim}{RGB}{20,105,176}
\colorlet{numb}{magenta!60!black}
\lstdefinelanguage{json}{
    basicstyle=\scriptsize\ttfamily,
    numbers=left,
    numberstyle=\scriptsize,
    stepnumber=1,
    numbersep=8pt,
    showstringspaces=false,
    breaklines=true,
    frame=lines,
    backgroundcolor=\color{background},
    literate=
     *{0}{{{\color{numb}0}}}{1}
      {1}{{{\color{numb}1}}}{1}
      {2}{{{\color{numb}2}}}{1}
      {3}{{{\color{numb}3}}}{1}
      {4}{{{\color{numb}4}}}{1}
      {5}{{{\color{numb}5}}}{1}
      {6}{{{\color{numb}6}}}{1}
      {7}{{{\color{numb}7}}}{1}
      {8}{{{\color{numb}8}}}{1}
      {9}{{{\color{numb}9}}}{1}
      {:}{{{\color{punct}{:}}}}{1}
      {,}{{{\color{punct}{,}}}}{1}
      {\{}{{{\color{delim}{\{}}}}{1}
      {\}}{{{\color{delim}{\}}}}}{1}
      {[}{{{\color{delim}{[}}}}{1}
      {]}{{{\color{delim}{]}}}}{1},
}

\begin{document}

\special{papersize=8.5in,11in}
\setlength{\pdfpageheight}{\paperheight}
\setlength{\pdfpagewidth}{\paperwidth}

\conferenceinfo{CONF 'yy}{Month d--d, 20yy, City, ST, Country} 
\copyrightyear{2015} 
\copyrightdata{978-1-nnnn-nnnn-n/yy/mm} 
\doi{nnnnnnn.nnnnnnn}

\title{Faster Interaction Through Lineage}

\authorinfo{David Tagatac\and Eugene Wu}
           {Columbia University}
           {\{dtagatac,ewu\}@cs.columbia.edu}

\maketitle

% \begin{abstract}
% TODO
% \end{abstract}

% \keywords
% keyword1, keyword2

\section{Introduction}
%TODO: more related work references
Scalability of big data visualization has been the subject of various proposed solutions.
The majority of these solutions focus on data reduction~\cite{Liu2013}, precomputation~\cite{Liu2013}, occlusion removal, and parallel computation~\cite{Liu2013}.
We propose a new optimization which uses indexed lineage annotations to reduce the number of queries required for certain interactions.
In order to accomodate these types of interactions, a new way of describing interactions is necessary.
This new way of describing interactions is declarative, like Vega~\cite{Satyanarayan}.
It differs from Vega, however, in that it focuses on the relationship between visualizations and tables.

\section{Describing Interactions}
\subsection{Connected Data Across Views Using SQL\href{http://randy.cs.columbia.edu/lineage/pgbench-connect/pgbench.html}{(\underline{link})}}\label{connect}
\begin{figure}[H]
	\includegraphics[width=\columnwidth]{figures/connect}
	\caption{A screenshot of the visualizations described in \autoref{connect}
	}
	\label{fig_connect}
\end{figure}
\subsubsection{Vega}
\subsubsection*{Scatter Plot (left)}
\lstinputlisting[language=json, tabsize=2, emptylines=0]{../pgbench-connect/vega/scatterplot.json}
\subsubsection*{Bar Chart (right)}
\lstinputlisting[language=json, tabsize=2, emptylines=0]{../pgbench-connect/vega/barchart.json}
\subsubsection{Alternative}
Let $V_1$ be the scatter plot.
Let $V_2$ be the bar chart.
\begin{align*}
	V_1&= \mathlarger{\pi}_{circle}(\bf{accounts})\\
	S_1&= \mathlarger{\sigma}_{P_1}(V_1)\\
	V_2&= \mathlarger{\pi}_{rect}(\mathlarger{\gamma}_{ccolor, \text{SUM}(abalance)}(\bf{accounts}\bowtie{}\bf{tellers}))\\
	S_2&= \mathlarger{\sigma}_{P_2}(V_2)\\
	init()&:\\
	&: P_1 = P_2 = *\\
	&: S_2.color = \texttt{steelblue}\\
	click(V_2)&:\\
	&: P_2 = "\text{WHERE}~ccolor=click.ccolor"\\
	&: S_2.color = red\\
	&: P_1 = "\text{WHERE}~aid~\text{IN~lineage}(S_2, \bf{accounts})"\\
	&: S_1.color = \texttt{red}\\
\end{align*}
\subsubsection{Comparison}
\subsection{Aggregation Filtering Using SQL\href{http://randy.cs.columbia.edu/lineage/pgbench-filter/pgbench.html}{(\underline{link})}}\label{filter}
\begin{figure}[H]
	\includegraphics[width=\columnwidth]{figures/filter}
	\caption{A screenshot of the visualizations described in \autoref{filter}
	}
	\label{fig_filter}
\end{figure}
\subsubsection{Vega}
\subsubsection*{Satisfaction Bar Chart (left)}
\lstinputlisting[language=json, tabsize=2, emptylines=0]{../pgbench-filter/vega/satisfactionbars.json}
\subsubsection*{Balance Bar Chart (right)}
\lstinputlisting[language=json, tabsize=2, emptylines=0]{../pgbench-filter/vega/balancebars.json}
\subsubsection{Alternative}
\subsubsection{Comparison}

\bibliographystyle{refs}
\bibliography{refs}

\end{document}