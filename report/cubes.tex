Here, we attempt to show that a data cube can be used to effectively index the relevant data for all interactions.
\subsection{Sketches}
\begin{claim}\label{alljoins}
A data cube can be used to effectively index all manners of joins.
\end{claim}
\begin{sketch}
Claim~\ref{alljoins} can be shown by induction.
\subsubsection*{Basis}
The following joins can be indexed effectively with data cubes:
\begin{enumerate}
	\item $X = A\bowtie{}B$
\end{enumerate}
\subsubsection*{Inductive Hypothesis}
Relation $H$ can be indexed effectively with a data cube.
\subsubsection*{Inductive Step}
In joining $H$ and another relation $I$, two possibilities arise.
\begin{enumerate}
	\item If the schemas of $H$ and $I$ do not share any common attributes, the data cube for $H$ must be expanded to include a new dimension for each attribute in $I$.
	Note, however, that this situation is unlikely to arise in DataViz as it results in a full cross product, and it does not really reveal anything new, information-theoretically.
	\item In the more common case, $H$ and $I$ share some attributes to be matched on the join.
	In this case, the data cube for $H$ must be expanded to include all \textit{new} attributes in $I$.
\end{enumerate}
In both cases, the new data cube effectively indexes the join $H\bowtie{}I$.
Noting that multijoins may be achieved commutatively, this inductive step describes joins of arbitrary complexity.
\end{sketch}
\subsection{Issues}
\begin{enumerate}
	\item What if an aggregation is joined with another relation?
\end{enumerate}